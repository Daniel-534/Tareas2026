\documentclass[a4paper]{article}

\usepackage[spanish]{babel}
\usepackage{graphicx}
\usepackage{amsmath, amssymb}
\usepackage[margin=2cm]{geometry}
\usepackage{fancyhdr}
\usepackage{enumerate}
\usepackage[shortlabels]{enumitem}
\usepackage{parskip}
\usepackage[most]{tcolorbox}
\usepackage[hidelinks]{hyperref}
\usepackage{float}


% cabecera
\pagestyle{fancy}
\fancyhead[l]{Daniel Soto}
\fancyhead[c]{Taller de Programación \#1}
\fancyhead[r]{\today}
\fancyfoot[c]{\thepage}
\renewcommand{\headrulewidth}{0.2pt} % linea horizontal

\begin{document}

\section{Definiciones y Ejemplos}

\subsection{Función suma de divisores $\sigma$}

\textbf{Definición:} Sea $n\in\mathbb{N}$, entonces

$$\sigma(n) = \sum_{d|n}d$$

\textbf{Ejemplo:}

$$\sigma(6) = 1+2+3+6 = 12$$

\subsection{Números Deficientes}
\textbf{Definición:} $n\in\mathbb{Z}^+$ es deficiente si

$$\sigma(n)<2n$$

\textbf{Ejemplo:} $n=8$ es deficiente porque

$$\sigma(8) = 1+2+4+8 = 15 <2\cdot 8 = 16$$

\subsection{Números Perfectos}

\textbf{Definición:} Un número $n\in\mathbb{N}$ es perfecto si

$$\sigma(n) = 2n$$

\textbf{Ejemplo:} $n=28$ es perfecto porque

$$\sigma(28) = 1+2+4+7+14+28 = 56 = 2\cdot 28$$

\subsection{Números abundantes}

\textbf{Definición:} $n\in\mathbb{Z}^+$ es abundante si

$$\sigma(n)>2n$$

\textbf{Ejemplo:} $n = 12$ es abundante porque

$$\sigma(12) = 1+2+3+4+6+12 = 28 >2\cdot 12 = 24$$

\section{Problemas}

\begin{enumerate}
    \item Construya la función $\sigma(n)$ utilizando \texttt{def}
    \item Construya funciones para determinar si $n$ es \textit{deficiente}, \textit{perfecto} o \textit{abundante}.
    \item ¿Por qué razón cree que a este tipo de números se los clasifica con estos nombres?
    \item Haga un \textit{plot} tipo \textit{scatter} de $(n, \sigma(n))$ tomando $n$ en el dominio que desee. Pinte los números deficientes con rojo, los números perfectos con verde y los números abundantes con azul.
    \item después de haber hecho y visto el plot, vuelva a decir por qué razón se les asigna este nombre a tales conjuntos de números. Compare esta respuesta con la respuesta a la pregunta $3)$
    \item Use la IA para construír 5 preguntas sobre lo hecho anteriormente donde se use estadística para llegar a respuestas interesantes. Deben ser 2 problemas fáciles, 2 nivel medio y 1 difícil.
    \begin{enumerate}
        \item \textbf{Nota 1:} Use \texttt{scipy.stats} para solucionar los problemas
        \item \textbf{Nota 2:} Use el modelo \textit{Qwen3-Coder} para construír las preguntas
        \item \textbf{Nota 3:} Asegúrese de especificar en el prompt que no le solucione los problemas
    \end{enumerate}
\end{enumerate}

\textbf{Recomendaciones respecto a la solución de los problemas:}

\begin{itemize}
    \item Trate de resolver los problemas en Python sin ayuda de la IA.
    \item Si no conoce los conceptos estadísticos que se presentan en los problemas, entonces antes de recurrir a la IA considere las siguientes opciones en orden:
    \begin{itemize}
        \item Busque en un libro de estadística (si lo conoce)
        \item Busque en Stack Overflow
        \item Busque directamente en internet
        \item Busque la respuesta, con explicación en algún chat de IA. Recomiendo Claude o Deepseek
    \end{itemize}
\end{itemize}

Al terminar de solucionar los 5 problemas, presente la solución en un notebook tipo \texttt{.ipynb}. Lo puede construír de manera remota con \textit{Google Colaboratory} o de manera local con \textit{Jupyter Notebook}.

\textbf{Nota:} Si necesita asesoría en algún punto de la actividad, no dude en preguntar.

\textbf{Contacto 1:} daniel.soto.villada@stud.uni-giessen.de

\textbf{Contacto 2:} daniel.sotov@upb.edu.co

\section{Recomendaciones:}

\begin{itemize}
    \item Tenga en cuenta el \texttt{PEP8}
    \item Escriba código autorreferenciado
    \item Escriba el código con el estilo \textit{camel case} o \textit{snake case}, no mezcle estilos
    \item Escriba el código en inglés
\end{itemize}
\end{document}
